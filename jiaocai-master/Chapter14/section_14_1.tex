\section{物联网安全概述}
\label{xdtz}

\subsection{什么是物联网?}
\label{IOT}

物联网其英文名称是:“Internet of things(IoT)”,从名字上看就是物物相联的互联网。关于物联网的起源,业界有不同的说法,其中一种说法是:最早的物联网设备是1991年在英国剑桥大学诞生的“特洛伊咖啡壶”。1991年,剑桥大学特洛伊计算机实验室的科学家们,经常要下楼去煮咖啡,还要时刻关注咖啡煮好了没有,既麻烦又耽误工作,于是,他们编写了一套程序,在咖啡壶旁边安装了一个便携式摄像头,利用终端计算机的图像捕捉技术,以3帧/秒的速率传递到实验室的计算机上。这样,工作人员就可以随时查看咖啡是否煮好,这就是物联网最早的雏形。

明确的物联网的概念是1999年美国麻省理工学院(MIT)的Kevin Ash-ton教授首次提出的。Kevin Ashton教授在1998年首次提出:“万物皆可通过网络相互联接”,首次阐明了物联网的基本含义,并且在麻省理工成立了“自动识别中心(Auto-ID)”,从此物联网有了基本雏形。Ashton在研究射频识别(RFID)和无线传感器网络(WSN)时提出的物联网概念,起初只是想通过RFID以及传感器技术让计算机对物理世界进行感知与识别,在无人干预的情况下汇聚数据,以此来对物品进行追踪和计数,他认为物联网与互联网一样有着改变世界的巨大潜力。


2008年,The Internet of Things in 2020报告中指出,物联网是由具有标示和虚拟个性的物体或对象所组成的网络,这些标示和个性等信息在智能空间中,使用智慧的接口与用户、社会和环境进行通信。

我国在2010年政府工作报告中,把物联网解释为:通过信息传感设备,按照约定的协议,把任何物品与互联网连接起来,进行信息交换和通信,以实现智能化识别、定位、跟踪、监控和管理的一种网络,它是在互联网基础上延伸和扩展的网络。

由此可见,物联网的概念是逐步发展和完善的,在以计算机为代表的第一次产业浪潮,以互联网、移动通信为代表的第二次产业浪潮之后,物联网正在引领信息产业的新浪潮,物联网产业蓬勃发展,极大的推动了经济发展和社会进步。


物联网技术让互联网不仅仅是延伸到最后1公里或者最后1英寸,还将无形的数字信号与有形的物理实体结合了起来。对于物联网来说,无论是作为主体还是客体,其实都可以把现实物理世界看作数字信息世界的一个直连组件。今天,物联网技术正在许多行业不断推广,并且为传统行业带来了巨大的变革。 

\textcolor{myblue}{\textbf{物联网赋能电网}}\quad 电力、天然气等公共事业公司派员工挨家挨户抄表的日子离我们越来越远了,如今的住宅都会安装一套分布式能源系统(Distributed Energy Resource, DER)系统,它可以分析电力需求和负荷数据并与配电网进行通信,进而识别出异常状态和不稳定情况,避免电压骤降和断电所付出的高昂代价。 

\textcolor{myblue}{\textbf{物联网赋能交通}}\quad
物联网已经给交通运输业带来了巨大改变,并且有望继续带来变革。\textcolor{black}{\textbf{智能交通系统}}(Intelligent Transportation System, ITS)可以发送排队预警信息,让车辆和驾驶员知道当前路段是否拥堵,然后车辆的导航系统可以针对拥堵路段迅速重新规划行驶路线,进而缓解拥堵状况。

\textcolor{myblue}{\textbf{物联网赋能智能制造}}\quad
“工业4.0”相信大家已经有所耳闻了,它主要用于描述通过自动化和数据交换来赋能智慧工厂的信息物理系统,其中涉及的用例包括远程监控、智能能耗管理、预测维护以及人机操作等,从而提高生产的灵活性。

\textcolor{myblue}{\textbf{物联网赋能智慧城市}}\quad
智慧城市是复杂物联网的一个实例,德勤最新发布的一份《超级智慧城市报告》(Super Smart City:Happier Society with Higher Quality)表示,目前全球已启动或在建的智慧城市已达1000多个,而中国在建数量大约500个,远超排名第二的欧洲(90个)。智慧城市系统通过安装若干个多功能传感器,对包括温度、气压、光线、一氧化碳、环境声强、行人和车辆的交通状况等信息进行测量,这些信息可以帮助城市管理者进行科学决策。


\subsection{物联网安全事件}
\label{computer}
任何事物都有其两面性,当我们探索物联网产业的发展机遇之时,却往往忽视了其背后的安全难题。截止到目前,因物联网设备自身漏洞被黑客攻击导致信息泄露或无法正常运行的事件依然频发,基于物联网终端的攻击事件不断见诸报端,物联网安全形势依然严峻。

2017年11月Check Point研究人员表示LG智能家居设备存在漏洞,黑客可以利用该漏洞完全控制一个用户账户,然后远程劫持LG SmartThinQ家用电器,包括冰箱,干衣机,洗碗机,微波炉以及吸尘机器人,并将它们转换为实时监控设备。2020年9月,安全公司 Avast 的研究员 Martin Hron 逆向了Smarter 公司的联网智能咖啡机,发现了很多漏洞。利用这些漏洞,他能触发咖啡机打开燃烧器、出水、启动磨豆机、显示勒索信息,并让咖啡机重复发出哔哔的声音,唯一阻止这一切混乱的方法是拔掉电源插头。Hron 发现咖啡机与智能手机应用之间的连接是没有加密的,没有身份验证,没有代码验证,最新的固件就储存在手机应用,可以很容易导出进行逆向工程。

除了家用物联网设备,车载驾驶辅助系统的安全性、可靠性也仍然未能让我们感到安心。比利时鲁汶大学(Belgian university KU Leuven)安全研究员列纳特·沃特斯(Lennert Wouters)表示,他只需大约90秒的时间,即可进入特斯拉汽车。进入车内大概1分钟左右的时间,他就可以注册自己的汽车钥匙,然后把车开走。物联网汽车不仅有被盗的风险,在驾驶中的安全性漏洞也得到了证实。《连线》杂志报道称,位于以色列内盖夫的本·古里安大学的研究人员已经证明,通过被劫持的互联网广告牌,就可诱导特斯拉电动汽车的 Autopilot 等系统采取突然的制动、驻车或转弯等操作。具体说来是,这些图像可能让自动驾驶/ 驾驶辅助系统紊乱,通过营造瞬间的光线投射,让系统“看到”不存在的物体(比如停车指示牌),从而让车辆系统做出反应。如果在恰当的时机实施这种攻击,会对车辆以及驾乘人员的安全造成严重威胁。

从提出“智慧城市”概念至今,在短短的10年内,智慧城市已经遍地开花。近日,IBM发布了一份网络安全白皮书,公布了其X-Force Red团队与安全业者Threatcare对智能城市所使用的主要系统的调查结果:共发现了17个安全漏洞,更严重的是,黑客可以通过这些漏洞任意摆布水位传感器,以触发错误的洪水警报,或是避免触发洪水警报,同样地,黑客也能操纵核电厂附近的辐射传感器,或者是借助对交通系统、建筑物警报系统或紧急警报系统的控制来制造混乱,这很可能对缺乏安全防范体系的智慧城市造成严重伤害。

以上的案例共同反映出了一个问题,那就是虽然物联网将许多的“不可能”变为了“可能”,但是在物联网产业建设初期,安全,作为物联网应用最基础保障,其防护体系的建设仍然任重道远。


\subsection{物联网安全特点}
\label{computer}

与传统意义上的网络安全不同,物联网安全是网络安全与其他工程学科相融合的产物。相比于单纯的数据、服务器、网络基础架构和信息安全,物联网安全的内涵要更加丰富。通常情况下,“网络安全”并不关注硬件设备以及同硬件设备交互的现实物理世界的物理安全和信息安全隐患,而正是通过网络对物理世界处理流程的数字化控制,使得物联网安全与传统安全有所区别,即物联网安全不再仅仅局限于包括机密性、完整性、不可否认性等基本信息安全保证原则,还需要对现实世界中收发信息的实体资源和设备进行安全保障。

物联网的安全特点主要体现在以下3个方面:

(1)安全体系结构复杂

物联网海量的感知终端,使其面临复杂的信任接入问题;物联网传输介质和方法的多样性,使其通信安全问题更加复杂;物联网感知的海量数据需要储存和保存,这使得数据安全变得十分重要。因此,构建合适全面、可靠传输和智能处理的物联网安全安全体系是物联网发展的一项重要工作。

(2)安全领域覆盖广泛

物联网所对应的传感器网络的数量以及其连接的智能终端规模巨大,这种体量是单个物联网传感器是无法比拟的,这会引入更复杂的访问控制问题;其次,物联网所连接的终端设备或器件的处理能力有很大差异,他们之间会相互作用,信任关系复杂,需要考虑差异化系统的安全问题;最后,物联网所处理的数据量将比现在的互联网大很多,需要考虑复杂的数据安全问题。

(3)有别于传统的信息安全

与传统意义上的网络安全不同,物联网安全是网络安全与其他工程学科相互融合的产物,相比与单纯的数据、服务器、网络基础架构的信息安全,物联网安全的内涵要更加丰富。在物联网环境中,即使分别确保了物联网各个层次的安全,也不能保证物联网整体的安全,这是因为物联网是融合多个层次于一体的大系统,许多安全问题来源于系统整合。例如,物联网的数据共享对安全性提出了更高的要求,物联网的应用需求对安全提出了新的挑战,物联网的用户终端对隐私保护的要求也日益复杂。

鉴于以上的安全特点,物联网的安全体系需要在现有信息安全体系之上,制定可持续发展的安全架构,使物联网在发展和应用过程中,其安全防护措施能够不断完善。

\subsection{物联网安全体系结构}
\label{computer}

在物联网系统中,主要的安全威胁来自以下几个方面:物联网传感器节点接入过程中的安全威胁、数据传输过程中的安全威胁、物联网数据处理过程中的安全威胁、物联网应用过程中的安全威胁等。这些威胁时全方位的,不管安全威胁的来源是哪些,我们都可以将物联网的安全需求归结为以下几个方面:物联网感知安全、物联网接入安全、物联网通信安全、物联网数据安全、物联网隐私安全。根据以上的需求,我们可以构建一个以需求驱动的物联网安全体系,具体包括以下内容:
\subsubsection{\textcolor{myblue}{\textbf{1. 物联网感知安全 }}}

物联网感知层的核心技术涉及传感器、条形码和RFID等技术,传感器在输出电信号时,容易受到外界干扰甚至破坏,从而导致感知数据错误、物联网系统工作异常等问题。例如,RFID技术使用电磁波进行通信,并且储存着大量数据,攻击者有可能通过窃听电磁波信号“偷听”传输内容,进而达到各种非法目的。

显然,物联网的感知节点接入和用户接入离不开身份认证、访问控制、数据加密和安全协议等信息安全技术。

\subsubsection{\textcolor{myblue}{\textbf{2. 物联网接入安全 }}}

在接入安全中,感知层的接入安全是重点。一个感知节点不能被未经认证授权的节点或系统访问,这涉及到节点的信任管理、身份认证、访问控制等方面的安全需求。另外,由于传感器节点受到功率和功能的制约,其安全保护机制交叉,其中的消息和数据传输协议没有统一的标准,从而无法提供一个统一、完善的安全保护体系。因此,传感器网络除了可能遭受与现有互联网相同的安全威胁外,还可能会受到恶意节点的攻击、传输的数据被监听或破坏、数据的一致性差等安全威胁。

另外,物联网授权要求信息在接入和传输过程中保证完整性,未经授权不能改变。即信息在存储或传输的过程中不被偶然或蓄意删除、篡改、伪造、乱序、重放等操作造成破坏和丢失。这时候还需要利用数字签名、加密传输等技术手段保持信息的正确生成、存储和传输。

\subsubsection{\textcolor{myblue}{\textbf{3. 物联网通信安全 }}}

物联网依靠通信终端来获取数据,目前的智能终端数量呈指数增长,而现有的通信网络承载能力有限,当大量的网络终端节点接入现有网络时,将会给通信网络带来更多的安全威胁。首先,大量终端节点的接入肯定会造成网络拥塞问题,给攻击者可乘之机。其次,由于物联网中设备传输的数量较小,为了降低功耗提高效率,很多通信协议都是明文传输(MQTT协议等),可能导致数据在传输过程中遭到攻击和破坏。

\subsubsection{\textcolor{myblue}{\textbf{4. 物联网数据安全 }}}

随着物联网的发展和普及,物联网数据呈爆炸式增长,在这种情况下,云计算、大数据等技术应运而生。虽然这些新型计算模式解决了个人和组织的数据处理需求问题,但同时,也增加了数据失控的危险。因此,针对数据处理中的安全隐私保护技术显得尤为重要。物联网安全审计要求物联网具有保密性和完整性,保密性要求信息不能泄露给未授权的用户,完整性要求信息不受各种破坏。

\subsubsection{\textcolor{myblue}{\textbf{5. 物联网隐私安全 }}}

除了上述安全指标之外,物联网中还需要考虑隐私安全问题。当今社会,无论是公众人物还是普通人,保护个人隐私已经成为了广泛共识。根据国家保密局发布的信息显示,物联网直接暴露于互联网,很容易遭受网络攻击,近20\%的企业或机构在过去的3年内遭受了至少一次物联网攻击,用户隐私也因此遭受着较大的泄露风险。在物联网隐私问题中,包含用户的信息隐私和位置隐私两个方面,信息隐私问题主要是指物联网中数据采集、传输和处理等过程中的秘密信息泄露,它往往与数据安全密不可分,因此一些信息隐私威胁可以通过数据安全的方法解决。位置隐私是物联网隐私保护的重要内容,主要指物联网中各节点的位置隐私以及物联网在提供各种位置服务时面临的位置泄露问题,具体包括RFID阅读器位置隐私、RFID用户位置隐私、传感器节点位置隐私以及基于位置服务中的位置隐私问题。


\subsection{物联网安全挑战}
\label{computer}

与传统网络相比,物联网发展带来的信息安全、网络安全、数据安全乃至国家安全问题将更加突出,我们要强化安全意识,把安全放在首位,超前研究物联网产业发展可能带来的安全问题。物联网除了要解决传统的安全问题之外,还需要克服认证接入、隐私保护等新的挑战,具体介绍如下:

\subsubsection{\textcolor{myblue}{\textbf{1.针对物联网的攻击增加}}}

卡巴斯基安全专家 Dan Demeter 表示:“由于物联网设备的发展,从智能手表到智能生活配件,已经成为我们日常生活中必不可少的一部分,网络犯罪分子巧妙地将注意力转移到了这一领域。” 由于物联网连接更多的智能终端,也会获取更多的隐私数据,相比互联网来说更加深入我们的生活,所以物联网与互联网的攻击方式有所区别。

本章\textcolor{myblue}{\textbf{第二节:针对物联网的特定攻击}}将结合物联网的特点,从物理设备层面和系统固件层面详细介绍针对物联网的攻击种类、效果以及造成的危害,最后介绍了针对一些特定攻击的前沿研究成果。

\subsubsection{\textcolor{myblue}{\textbf{2. 物联网协议安全性低}}}

目前物联网通信协议仍然存在许多严重的安全问题\cite{ref1},有些研究\cite{ref2,ref3,ref4}表明了物联网协议栈中链路层存在安全漏洞,攻击者可以发起时间同步攻击、篡改攻击和能量耗尽攻击等。也有研究者\cite{ref5,ref6}指出了物联网协议栈中 RPL协议存在多种 安全漏洞,攻击者可以发起女巫攻击、选择前向攻击、HELLO 洪泛攻击以及虫洞攻击等,从而破坏 RPL协议正常运行。

本章\textcolor{myblue}{\textbf{第三节:物联网网络协议安全}}将介绍现有的物联网通信协议,分析现有协议的用途以及其安全缺陷,介绍目前的前沿研究方向。

\subsubsection{\textcolor{myblue}{\textbf{3.物联网设备的认证与授权机制不完善}}}

Zscaler公司最近发布了其第二份年度物联网报告,报告中对212个制造商的21种不同的共计553种IoT设备进行了统计分析。其中最让人担忧的是“影子物联网”的存在,世界各地的企业、机构都在观察这种影子物联网现象,即员工将未经授权的设备带入企业,连接网络进行办公,这些未知和未经授权的设备给网络系统带来了巨大的安全隐患。报告中还指出,未经授权的物联网设备数量上升趋势明显,针对未经授权设备的新攻击不断涌现。

本章\textcolor{myblue}{\textbf{第四节:物联网设备的认证与授权}}将介绍现有的物联网设备的认证与授权方案,围绕非法接入和非法访问分析现有方案的安全缺陷,然后介绍目前的研究方向以及研究成果。

\subsubsection{\textcolor{myblue}{\textbf{4.物联网隐私泄露问题堪忧}}}

近年来,随着GDPR(General Data Protection Regulation)等法律法规框架日益受到重视,以及网络威胁形势变得更加动态和复杂,如何更好地保护敏感数据和个人数据隐私的问题变得日益突出。物联网(IoT)正在改变多个行业,但同时,物联网需要使用智能终端等设备收集大量的数据并对其进行分析,所以不可避免的在数据隐私方面带来了一些挑战。

本章\textcolor{myblue}{\textbf{第五节:物联网隐私保护}}将介绍目前物联网隐私保护面临的新情况与新挑战,然后介绍物联网隐私数据的具体分类,最后介绍现有的物联网轻量级隐私保护技术与目前物联网隐私保护领域的最新成果。

