\section{物联网网络协议安全}
\label{protocol_security}

\par 在物联网系统中,通信设备依靠网络协议来提供服务,确保这些协议的安全性与可靠性势在必行。事实上,物联网协议安全对于物联网的存在与事物本身一样重要。但是,由于不同环境的功能差异以及标准操作系统的缺失,物联网生态系统面临着多种数据内容与形式,这一多样性使得开发适用于各种物联网设备和系统的标准安全协议变得困难。
\par 依照适用范围来看,物联网领域的通信协议主要分为以Bluetooth为代表的\textcolor{myblue}{\textbf{物联网通用网络协议}}和以NB-IoT为代表的\textcolor{myblue}{\textbf{物联网专用网络协议}}。本节主要介绍常见的物联网通信协议与安全机制,以及相关领域的前沿工作进展。

\subsection{物联网通用网络协议安全}
\label{common_security}

\subsubsection{\textcolor{myblue}{\textbf{1. Bluetooth}}}
\par 蓝牙(Bluetooth)是一种短距离无线通信标准,主要工作在2.4GHz的\textcolor{myblue}{\textbf{ISM频段}}\snote{该频段主要向工业,科学和医学领域开放,无需授权,免费使用},采用分散式网络结构以及快调频和短包技术,能够实现全双工通信。由于具有低功耗、低成本、抗干扰等特点,蓝牙耳机、蓝牙手环、车载蓝牙等设备无一不在影响人们的日常生活,蓝牙技术已广泛应用于语音和数据通信。
\par 蓝牙通过预先配对机制保证安全连接,只有在受信任的设备之间才允许数据交换。但蓝牙技术仍存在一些潜在的安全隐患\cite{hassan2018security},下文将分别介绍若干具有代表性的攻击方式。
\par \textcolor{myblue}{\textbf{Bluetracking}} \quad Bluetracking是一种基于蓝牙信号的追踪与定位技术。攻击者无法访问目标设备的任何内容,但可以跟踪蓝牙信号来定位受害者的家庭地址,也可以通过观察受害者的位置变化来预测行踪。
\par \textcolor{myblue}{\textbf{Bluejacking}}\quad Bluejacking通过蓝牙连接往其它设备强制发送信息。攻击者首先扫描蓝牙信号,对于打开蓝牙并且设置可见的设备,可以定时发送事先设置好的文本或图片。这种攻击不会窃取或改变设备的任何数据,多用于宣传广告游戏的游击营销活动。
\par \textcolor{myblue}{\textbf{Blueforce}}\quad Bruteforce旨在窃取蓝牙设备的\textcolor{myblue}{\textbf{MAC地址}}\snote{也称为物理地址,长度为48位,是制造商为网络硬件(如无线网卡)分配的唯一代码}。MAC地址的前24位通常是固定的,后24位大约有1680万种可能的组合,平均需要840万次尝试才能猜中。但通过使用智能工具包和免费的开源软件,很容易将它破译出来。一旦确定了受害者设备的MAC地址,攻击者就会将自己的MAC地址伪装成受害者的MAC地址,并窃听受害者的信息。
\par \textcolor{myblue}{\textbf{Bluesnarfing}}\quad Bluesnarfing是一种再未经用户同意的情况下非法访问蓝牙设备的攻击方法。攻击人通过与目标手机建立非法连接,可以监听电话,窃取联系人,乃至阅读和发送信息。它的一个进阶版本被称为Bluesnarfing++。该方法允许黑客访问设备中的许多功能和数据,将来电或短信转移到其他设备上。
\par 从1.0到5.0,蓝牙技术不断推陈出新,应用场景越来越广阔,安全问题也日益得到重视。企业和厂商应加强蓝牙设备配对和连接环节的安全考量,而非“打地鼠”式地被动解决问题。比如在配对时,增加验证配对密钥环节;在连接时,使用高级的相互身份验证方式;硬件上,可采用高安全性的蓝牙系统芯片和模块,减少安全漏洞带给用户的影响。另外,由于蓝牙系统安全性能遵循“短板效应”,用户应当及时更新蓝牙系统版本,避免因协议落后引发的安全漏洞造成个人信息泄露与财产损失。

\subsubsection{\textcolor{myblue}{\textbf{2. Wi-Fi}}}
\par Wi-Fi是一种基于IEEE 802.11标准的无线局域网技术,采用星型拓扑结构。与有线网络技术相比,它具有使用灵活、建网迅速、个性化高等特点,能够方便灵活地为用户提供网络接入。在互联网进入千家万户的智能时代,Wi-Fi与我们的生活早已密不可分。
\par 在2019年的“3·15”晚会上,央视曝光了一种基于Wi-Fi的攻击手段。只要用户手机开启Wi-Fi功能,探针盒子就可以获取手机的MAC地址,通过与后台的数据库进行匹配,不仅可以将其转换为手机号,还可以对用户进行精准画像,获取用户性别、年龄、常用APP等敏感信息。在万物互联的时代,诸如Wi-Fi探针的安全问题屡见不鲜。在实现高效沟通的同时,我们也在无形之中让渡了个人隐私。作为Wi-Fi使用者,应当谨慎连接未知网络,一旦连接到黑客设定的热点,所有数据包都会经过特定设备转发,黑客将这些信息被嗅探下来进行分析,就可以获得通信记录;作为Wi-Fi管理者,应当采取使用WPA/WPA2加密方式,关闭远程管理端口等措施,防止被恶意蹭网。
 
\subsubsection{\textcolor{myblue}{\textbf{3. NFC}}}
\par NFC(Near Field Communication,近场通信)由射频识别技术(RFID)演变而来,是一种近距离的高频无线通信协议,可用于电子设备之间非接触式的点对点传输数据。NFC让消费者简单快捷地交换信息与访问服务,自2003年问世以来,凭借其安全性与便捷性得到了众多企业的青睐。如今,NFC技术已经广泛应用于移动支付、电子票务、门禁识别等领域。
\par 尽管NFC的有效触发距离较短(通常小于10cm),但由于它不需要来自用户端的额外认证,在交易时仍然存在安全隐患。除了实卡盗刷、损坏等传统的攻击手段,虫洞攻击\cite{giese2019security}也逐渐进入公众视野。使用一台NFCGate服务器,攻击者只需要站在受害者身边,就能实现近距离地非接触式支付。在地铁、公交等人流密集场所,这种攻击变得难以预防。事实上,攻击者的同伙可以在世界各地使用受害者的的支付方式付款。

\subsection{物联网专用网络协议安全}
\label{specific_security}

\subsubsection{\textcolor{myblue}{\textbf{1. ZigBee}}}
\par 蜜蜂在发现花丛后会通过一种抖动翅膀的“舞蹈”向同伴传递信息,受此启发,人们设计了一种类似结构的通信协议——ZigBee。ZigBee是一种基于IEEE802.15.4标准规范的短距离、低功耗的无线通信技术,具有成本低廉、操作简单、连接稳定等特点,最初主要为工业领域自动化的数据传输而建立。由于它的优势与智能家居的要求不谋而合,ZigBee在该领域也逐渐大放异彩。
\par ZigBee的安全策略主要由网络层和应用支持子层提供,主要包括安全密钥建立、安全密钥传输、对称加密的帧保护以及安全设备管理等模块。为了保持设备的低成本,低能耗和高兼容性,ZigBee必须在安全性与便携性之间做出权衡。它使用预先安装的密钥进行设备的安全配置,当一个新设备加入网络时,黑客将有机会从外部嗅探出网络的交换密钥,从而接管该网络内所有设备的控制权;此外,ZigBee采用“开放信任”模型。协议栈层相互信任,仅由发起帧的层负责对其进行保护,这也容易出现安全问题。

\subsubsection{\textcolor{myblue}{\textbf{2. MQTT}}}
\par MQTT(Message Queuing Telemetry Transport,消息队列遥测传输)基于TCP/IP协议栈而构建,是一种轻量而灵活的网络协议,可在受限的设备硬件和高延迟的网络上实现。MQTT的一个关键特性是发布和订阅模式,允许使用一台服务器连接成千上万的客户端。尽管MQTT已经成为事实上的数据传输标准,但是仍存在一些安全问题\cite{chen2020a}。
\par \textcolor{myblue}{\textbf{重放攻击(replay attack)}}\quad 重放攻击是指攻击者发送一个目的主机已经接收的包,来达到欺骗系统的目的,主要用于身份认证过程,破坏认证的正确性。可以通过加随机数、时间戳等方案加以应对,保证数据流的单向性。
\par \textcolor{myblue}{\textbf{中间人攻击(man-in-the-middle attack)}}\quad 中间人攻击是指介入网络通信过程并修改通讯信息,需要攻击者对网络协议有较为深入的了解。在一个典型的攻击场景中,可以使用基于BERT的对抗模型生成恶意消息,并通过一个解析器分析与篡改MQTT消息\cite{wong2020man}。由于MQTT协议在通信传输过程中缺乏真实的信源认证,仅采用传统的信道安全技术无法屏蔽中间人攻击。

\subsubsection{\textcolor{myblue}{\textbf{3. LoRa}}}
\par 随着智慧城市、智能家居概念兴起与相关应用逐渐落地,作为低功耗广域网(LPWAN)中的一种长距离通信技术,LoRa受到越来越多的青睐。它一改以往关于传输距离与通信功耗的折中方案,为用户提供了一种远距离、长寿命、大容量的传感网络。因其功耗低、距离远、组网灵活等诸多特性与物联网碎片化、低成本、大连接的需求高度契合,Lora被广泛部署在智慧社区、智能家居等多个垂直行业,具有广阔的发展前景。
\par 与许多通信协议一样,LoRa也面临着诸如窃听、选择性转发、节点冒充等安全隐患\cite{zhou2008securing}。LoRa利用AES-128位的密钥机制保障系统安全。即便如此,它在安全性上依然面临诸多问题和挑战。首先,密钥管理薄弱且存在风险,LoRa的网络层和应用层是由相同的根密钥和随机数生成的,并且这两层密钥并不独立,加密强度不够,存在由于私钥泄漏引起的数据隐私泄漏和数据篡改风险;其二,终端网络认证不具备安全存储介质,其安全性仅取决于终端的物理保护,较弱的终端存在更大的泄漏风险;此外,在网络服务器上生成的应用和网络会话密钥将在网络和应用服务提供者之间产生利益冲突。网络服务器和应用服务器可以同时导出应用和网络会话密钥,如果这一过程涉及两个不同的组织,密钥的安全性就难以得到保证。为此,可以利用可信第三方的密钥管理架构,利用无线传感器网络来解决密钥管理和更新机制的问题\cite{seo2014effective,agrawal2012a}。但由于引入了第三方系统,通信开销也随之增加。

\subsubsection{\textcolor{myblue}{\textbf{4. NB-IoT}}}
\par NB-IoT(Narrow Band Internet of Things)是一种新兴广域网技术,支持低功耗设备在广域网的蜂窝数据连接,具有多链接、广覆盖、低速率、低成本等特点。NB-IoT占用180kHz带宽,包含带内部署、保护带部署、独立部署三种部署方式。尽管NB-IoT极大地便利了物联网系统的部署方式,但该协议并没有设计标准化的安全架构。下文介绍若干基于资源分配的攻击手段。
\par \textcolor{myblue}{\textbf{IP欺骗(IP sproofing)}}\quad IP欺骗是DDoS中较为常见的攻击类型\cite{rajashree2018security}。它是指攻击者伪造自身的 IP 地址向目标系统发送恶意请求,造成目标系统受到攻击却无法确认攻击源,或者取得目标系统的信任以获取机密信息。
\par \textcolor{myblue}{\textbf{洪泛攻击(flooding attack)}}\quad 洪泛攻击是指窃听者向设备传输大量的请求,人为制造系统繁忙,导致设备无法响应来自合法用户的请求。
\par \textcolor{myblue}{\textbf{带宽欺骗(bandwidth sproofing)}}\quad 带宽欺骗是指当系统带宽被分配给NB-IoT设备时,攻击者更容易得到带宽。由于NB-IoT设备工作在非常低的带宽上,与其他技术相比,这种类型的攻击较为普遍。对此,有学者利用博弈论思想,设计了一个自适应的入侵检测系统以消除恶意干扰,保持网络组件的安全性\cite{supply2018bandwidth}。
\par 在支持NB-IoT的设备中,视频录像机和网络摄像机往往首当其冲。2017年,趋势科技公司的一份报告指出,大约有12万台网络摄像机面临名为Persirai的僵尸网络感染风险。由于所有用户都可以通过广域网远程访问摄像头,这些设备就会赤裸裸地暴露在网络上。通过访问这些设备中易受攻击的接口,攻击者就能注入命令,强制连接至某个站点,并下载执行恶意脚本。默认密码、弱口令都是黑客发起攻击的突破口,攻击者通过密码组合进行大规模地自动化入侵,从而接管网络内的设备。除了定期修改并设置高安全性的密码之外,网络摄像头的管理人员也应当禁用路由器上的通用即插即用功能,以防止网络设备在无需任何警告的情况下,就可以实现端口映射。此外,设备制造商在生产环节也应提高安全意识,及时修复远程访问等环节的安全漏洞,从源头进行预防。

\subsection{前沿研究}
\label{latest_research}
\par 安全通信、认证访问和信任管理,一直是物联网协议研究的核心命题,而协议轻量化是目前物联网安全研究的显著特点。最近的研究表明,双重认证密钥交换协议对普通节点来说是不够安全的,它降低了计算的复杂性\cite{shin2018two-factor}。为此,6LowPSec\cite{glissa20196lowpsec,qiu2016a}使用混合加密法实现各个节点信息之间的安全交换,在降低计算与传输开销的同时,成功防止各类恶意攻击,为6LoWPAN协议的端到端通信提供了一个轻量级的相互认证与密钥建立方案。基于社交网络的轻量级认证协议SNAuth\cite{dao2017achievable}利用微控制器单元(MCU)平台,以支持对等意识通信(peer-aware communication)设备之间的认证与密钥协商过程,有效防止数据篡改风险。在MQTT领域,一种名为“模糊(fuzzing)”的测试技术,通过向目标系统发送意外的输入数据,发现了若干迄今未曾报道过的安全漏洞\cite{hernandez2018mqtt},可见其作为漏洞挖掘工具的实用性。在RFID领域,CDTA改进了读取器和标签之间的相互认证方式,以实现假货检测与设备追踪的目标\cite{yang2017a}。
\par 在硬件层面,也涌现出一批效果卓越的解决方案。有学者\cite{lee2021deep}基于深度学习方法,利用通信设备的物理特征——射频指纹来对NFC标签进行识别,以此来防止克隆攻击。针对数据损坏和拒绝服务攻击,ECC和AES算法是构建安全通信通道时的最佳抵御技术\cite{singh2018near},可以通过重组数据路径与密钥扩展,减少寄存器与控制逻辑的数量,来实现硬件优化\cite{bui2017aes}。还有一种基于IEEE802.15.4的信号收发器结构\cite{nain2017a},它使用物理层加密方法,能够在不影响合法接收器的性能下,给恶意节点带来高错误率,并通过增加运算复杂性来提供对流量耗尽攻击的保护。
\par 时间同步也是物联网中速度计算、电源管理等模块的一个关键功能。为了防止恶意节点通过广播虚假时间戳来降低整个网络的准确性,可以使用一种用于大规模物联网的安全时间同步模型\cite{qiu2017a}。在该模型中,一个节点利用其父节点和祖父节点来检测恶意节点,并逐步构建一个与参考节点同步的生成树拓扑。Bio-Hashing技术将该方案进一步拓展到了动态IP领域\cite{tao2018accessauth}。
