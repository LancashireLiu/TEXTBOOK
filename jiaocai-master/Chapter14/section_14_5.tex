\section{物联网隐私保护技术}
\label{protocol_security}
随着物联网隐私安全引起越来越多的关注,相应的隐私保护技术也在快速发展。
传统隐私保护技术主要利用密码学方法实施端到端加密以保护不安全的通信,从而保护用户隐私,如同态加密、安全多方计算等。而物联网隐私保护由于其资源限制大、网络节点数目多等特点无法直接使用传统隐私保护技术,需要发展适用于物联网的隐私保护技术。


本节将从以下几个方面介绍目前物联网隐私保护技术的发展情况:首先介绍目前物联网隐私保护面临的新情况与新挑战,然后介绍物联网隐私数据的具体分类,最后介绍现有的物联网轻量级隐私保护技术与目前物联网隐私保护领域的最新成果。


\subsection{物联网隐私保护新特点}
\label{common_security}

随着物联网的应用,物联网在方便用户的同时,也记录着与用户隐私相关的大量数据,这使得数据隐私安全问题迫在眉睫。2015年至今,国内外发生多起智能玩具、智能手表等漏洞攻击事件,超百万家庭和儿童信息、对话录音信息、行动轨迹信息等被泄露;我国某安防公司制造的物联网摄像头存在多个漏洞,黑客仅凭默认密码登录设备即可访问摄像头的实时画面。此外,据有关数据显示,10000户家庭每天大约能生成多达1.5亿个离散数据点。IDC报告显示,2020年全球物联网设备将有200亿~250亿台。庞大的物联网设备规模意味着有海量用户的不计其数的隐私数据被收集记录,隐私数据泄漏的风险系数被急剧放大。因此,在万物互联把数据的收集和分析的触角延伸到人们生活的各种场景同时,安全和个人数据的保护不善威胁着物理终端设备背后人们的真实生活和生命财产。与传统互联网相比较,物联网的安全问题更加复杂。在IoT场景下,我们面临的三大安全和隐私挑战是:终端认证,网络攻击预防和家庭个人数据保护。这三大挑战分别着眼于物联网“端、网、云”的三层网络架构~\cite{wenti2019}:
\par1.从“端”来看,不同物联网终端设备的安全防护能力参差不齐,差异较大。在物联网中,终端设备负责感知环境信息,例如感应、识别物体,获取环境湿度、温度等,由此产生了种类繁多的物联网设备,包括RFID芯片、指纹扫描器、温度湿度传感器、网络摄像头等。然而,大多终端设备应用场景简单,存储与计算能力有限,无法满足部署安全软件或者高复杂度的加解密算法的算力与存储需求。此外,物联网终端的‘移动’属性,消解了传统网络边界,使得依托于网络边界的安全产品无法正常发挥作用,如果物联网终端处在无人监控的环境中,则攻击者更容易对其实施攻击。
\par2.从“网”来看,物联网网络结构复杂且通信协议安全性差。相比互联网,物联网网络采用多种异构网络,更为复杂,存在算法破解、协议破解等风险以及中间人攻击等攻击威胁,Key、协议、核心算法、证书等暴力破解的情况时有发生。因此,物联网数据传输管道的防护安全与管道中的流量内容安全问题不容忽视。
\par3.从“云”来看,云平台层安全风险对于整个网络生态不容忽视。云平台通常通过网络汇集终端设备获取的感知数据,然后借助App与云平台进行信息交互,达到远程管理设备的目的。目前,云安全技术水平日益成熟,但更多的安全威胁往往来自内部的不当管理或外部渗透攻击技术。如果企业内部管理机制不完善、系统安全防护不配套,那企业内部管理过程逻辑漏洞就可能让平台或整个生态彻底沦陷。此外,依托于社会工程学的非传统网络攻击始终存在,这给系统的安全防护策略带来极大的挑战。\par

在这种物联网环境下,对于个人隐私数据的保护需要对数据进行科学分类,有效的数据分类方法有利于防护手段向灵活精细方向发展,促进系统整体防护能力的提升。\par
    

\subsection{物联网隐私数据分类}
\label{common_security}


\textcolor{myblue}{\textbf{隐私数据}}指个人或团体不愿意被他人获得的敏感信息。区别于普通数据,隐私数据常常由私人设备产生,如手机,平板电脑等,有时也可能由一些传感器产生,如摄像头等。物联网隐私数据数量庞大种类繁多\cite{qian2013},根据是否随时间变化可以分为\textcolor{myblue}{\textbf{动态数据}}和\textcolor{myblue}{\textbf{静态数据}}\cite{min2021};根据是否包含生物特征可以分为\textcolor{myblue}{\textbf{生物特征数据}}和\textcolor{myblue}{\textbf{非生物特征数据}}\cite{Preserving2019};根据能否确定一个人的身份可以分为\textcolor{myblue}{\textbf{个人数据}}和\textcolor{myblue}{\textbf{非个人数据}}\cite{Protecting2019}。


\subsubsection{\textcolor{myblue}{\textbf{1. 动态数据和静态数据}}}
\textcolor{myblue}{\textbf{动态数据}}(dynamic data):动态数据随着时间常常发生变化,如用户的实时位置。

\textcolor{myblue}{\textbf{静态数据}}(static data):静态数据指在一定时间段内不容易发生变化,如用户的的姓名、性别和家庭地址等。
\subsubsection{\textcolor{myblue}{\textbf{2. 生物特征数据和非生物特征数据}}}
\textcolor{myblue}{\textbf{生物特征数据}}(biometric data):生物特征数据指生物与生俱来与众不同的行为和特征,如声音特征、指纹、面部特征等,通常可以用来认证或识别用户的个人身份。生物特征数据一旦发生泄漏,在安全性方面可能变得毫无用处,因为它常常是不可更新的。

\textcolor{myblue}{\textbf{非生物特征数据}}(non-biometric data):除生物特征数据以外的数据为非生物特征数据,如说话时的内容、情绪,密码等。
\subsubsection{\textcolor{myblue}{\textbf{3. 个人数据和非个人数据}}}
\textcolor{myblue}{\textbf{个人数据}}(personal data):个人数据指可以唯一确定一个人身份的数据,如一个人的邮箱地址、手机号等。

\textcolor{myblue}{\textbf{非个人数据}}(non-personal data):不能确定一个人身份的数据。值得注意的是,当一些非个人数据组合在一起,有时也可以确定一个人的身份。如姓名和公司都不是个人数据,但组合在一起很大概率可以确定一个人。

针对这些隐私数据,目前已经提出了很多隐私保护方法,但在物联网中,资源常常是有限的,所以通常采用一些轻量级的方法。需要说明的是,现在的隐私保护方法不一定细化到针对各个类别的数据用不同的策略进行保护,但是为了开发出灵活有效的物联网轻量级保护方案,这是一个值得研究的方向。

 
 
\subsection{物联网轻量级隐私保护技术}
保护物联网隐私数据的常用解决方案是实施端到端加密以保护不安全的通信。
端到端加密可以分为非对称加密和对称加密。在非对称加密通信过程,当实体A尝试与另一实体B通信时,A利用B的公钥加密消息,而B利用其私钥进行解密。

公钥可以通过不可靠的通道传输,这大大方便了密钥的授权。
然而,基于非对称算法的加密和解密过程是计算密集型的,需要消耗大量的能源和计算资源\cite{luo}。
而资源受限是物联网的一大特点:由于物联网设备需与实际物品相结合,其体积受到较大限制,且大多数物联网设备需在无充电情况下维持长时间工作,
其用于加密计算的资源大大受限。
有研究表明非对称加密的计算消耗至少是对称密钥方案的100倍\cite{hirani}。
因此,对称加密是目前主流的物联网隐私保护技术轻量化发展方向。 


将对称加密用于物联网隐私保护面临的主要挑战是设计一种健壮的密钥交换机制。
当网络中某个节点被捕获时,与之通信节点的对称密钥也将泄露。
为了抵御节点捕获攻击,研究者提出了一种大规模物联网动态随机密钥建立机制,该机制支持传感器节点的多阶段部署,
通过动态分配密钥,在另一个部署阶段发生之前,部署阶段中已经部署的节点在密钥环中刷新自己的密钥\cite{das}。
另一种应对节点捕获攻击的方法是通过网络中所有节点协同生成全局密钥,该方法从网络末端节点开始生成密钥,向根节点传递并最终生成全局密钥,当有节点离开或加入网络时重新生成全局密钥\cite{hendaoui}。
而为了提高对称密钥生成效率,Kronecker积被用于分解对称矩阵,分解得到的因子矩阵中的参数分配至网络中各个节点作为密钥\cite{tsai}。


混沌系统也被用于轻量化物联网隐私保护技术。混沌系统对初始条件非常敏感,这意味着两个输入稍有不同的混沌系统的输出可能会显著不同,非常适合用于密钥生成。
此外,在基于混沌的密码系统中,即使恢复了其中一个密钥,攻击者恢复其他密钥的可能性也微乎其微,对节点捕获攻击有较强的抵抗能力。
研究者首先针对智能家居系统设计了基于混沌系统的对称加密机制,该机制利用混沌系统生成密钥,并通过消息认证码来进行节点设备认证并确保数据完整性\cite{song}。
近期,适用于大型物联网网络的基于混沌系统的对称加密机制被提出。

该机制首先利用混沌系统为每个物联网设备分配参数与初始值(即密钥)作为默认配置。
当设备首次连接到控制中心(如云平台)时,该参数和初始值用于设备注册认证。
之后,该控制中心为设备间通信分配另一对参数和初始值作为通信密钥,并使用特定参数和初始值迭代混沌系统,同步更新所有密钥\cite{luo}。


以上提出的轻量化物联网隐私保护方法仍面临只适用于小型网络或只适用于中心架构网络等问题,需要进一步的发展与完善。

\subsection{前沿研究}
\label{recentwork}

为了在物联网设备的有限资源下可靠实时地保护隐私数据,研究者们提出了诸多新的解决方案。一方面,为使得隐私保护更具实践性,研究者们在传统的物联网隐私保护技术基础上进行改进创新;另一方面,随着机器学习、区块链等技术的迅猛发展,这些技术被融入物联网中,为隐私数据提供更可靠的保障。

传统的物联网隐私保护技术包括数据匿名化、密码技术等。数据匿名化在用户和服务提供方之间引入了更多的匿名参与者以保护用户的轨迹和内容隐私,如k-匿名技术\cite{Zhang2017}。但使用数据匿名化技术保护隐私的代价是信息量的损失,而对隐私数据加解密却能无损地保护隐私。密码技术被广泛用于物联网和云服务器场景下的认证方案中以抵御离线密码猜测攻击,如椭圆曲线密码\cite{Kumari2018}。密码技术中,块密码及非对称密码的加解密速度慢,很难满足物联网设备对实时性的要求;而流密码由于其可在硬件上高效加解密的特性,从而在物联网设备中得到了广泛应用。一些流密码算法,如RCCM和PARCCM\cite{Liu2020},与边缘计算结合实现了高效地保护隐私。

物联网设备可以通过应用机器学习或深度学习算法保护隐私数据,这些算法将对接入点收集的信息进行处理,从而准确有效地鉴别出隐私威胁。例如,在VANET的场景下,车辆节点间的协作学习将会引发隐私问题,给予恶意节点根据观察到的信息推断敏感信息的机会。因此,分布式ML算法的协作IDS\cite{Tao2018}利用本地数据和共享知识检测恶意节点并保护隐私。和传统机器学习算法相比,深度学习在多项研究中都表现出了更优的速度和准确率。但无论是机器学习还是深度学习算法,算法自身会受到如对抗性攻击等威胁,这将为物联网隐私安全带来新的漏洞。


随着区块链在物联网中的融入,区块链的去中心化、信息不可篡改等优良特性将给予物联网隐私数据以更可靠的保护。区块链的去中心化特征可以帮助抵抗恶意参与者的操纵和伪造,而区块链的身份和访问管理系统将能更好地应对挑战并保护隐私数据\cite{Kshetri2017}。例如,在P2P数据共享车载网络场景下,由于车辆系统的资源限制,数据被转发至边缘计算机,此过程中恶意节点极易窃取隐私数据。因此,车载物联网数据共享方案\cite{Kang2018}利用联盟链的优势,只允许指定节点执行审计和验证的工作,并使用智能合约以确保用户的真实性。然而,仍存在一些问题导致区块链难以实际应用于物联网隐私保护中,如区块链对存储和计算资源要求高,速度慢,自身可能存在漏洞等等。
