\section{针对物联网的特定攻击}
\label{specific-attack}

物联网技术不断发展变革,深刻影响着传统产业形态和人们的生活方式。通过泛在对象设备化、自治终端互联化、普适服务智能化,物联网收集、存储、分析海量数据用于提供即时、便捷、普适的智能化服务,在智慧城市、智慧金融、智慧医疗、智慧农业、智慧交通等诸多领域展现出了无与伦比的发展前景。然而,随着数以亿计的设备接入物联网,加之终端设备自身异质性、分散性、脆弱性的特点,整个生态体系中针对物联网的特定攻击和欺骗行为大量涌现,严重威胁着物联网安全。一是在\textcolor{myblue}{\textbf{物理设备层面}},物联网具有开放包容的特性,各种异构终端自主互通互联、共享数据,丰富的设备类型和庞大的网络流量为侧信道攻击提供了可行条件,攻击者可通过各种隐匿的侧信道轻易窃听、篡改、伪造数据从而造成数据泄露和污染;二是在\textcolor{myblue}{\textbf{系统固件层面}},泛在部署的物联网设备受限于计算、存储和网络资源,针对系统固件缺乏强有力的安全保护机制和自适应的漏洞检测手段,因而设备本身的固件漏洞仍然是其遭受威胁的主要因素。本节将从物理层侧信道威胁和系统级固件脆弱性两方面切入,综合介绍针对物联网的特定攻击及其前沿研究。


\subsection{物理层侧信道威胁}
\label{side-channel}

物联网设备使用过程中,会接收或者发送各种模态的信号。例如,在智能音箱在接收端,需要分析语音信号中的控制指令;在发送端,需要播放声音信号。而麦克风与扬声器作为发送端与接收端的两种重要传感器,除了能够处理声音信号之外,还会泄露振动信号、电磁信号等其他模态的信号。

正是因为物联网设备会接收或者泄露多种模态的信号,出现了五花八门的侧信道攻击,这些攻击针对的不是物联网设备使用的主要信道,而是伴随主要信道产生的次要信道,从而实现信息注入或者读取的目的。例如,对于重要的输入设备“键盘”,出现了利用键盘敲击声音的推测出键盘敲击内容的侧信道攻击;对于智能语音助手的接收设备麦克风,出现了利用超声波注入语音指令的侧信道攻击;对于涉及WiFi信道的路由器,出现了利用WiFi信道窃听人体语音的侧信道攻击。

这些侧信道攻击往往难以预防,这是由于这些被攻击的物联网设备在制造之初,没有考虑侧信道攻击的威胁。对于键盘,从声音信道泄露的键盘敲击声不会被刻意消除;对于智能语音助手的麦克风,从超声波信道注入的语音命令难以被滤除;对于本来用于传递数据信息路由器,从WiFi信道中感知人体语音信息的行为难以被阻止。


\subsubsection{\textcolor{myblue}{\textbf{1. 基于机械波的攻击 }}}

在物联网系统中,智能语音设备处于控制中心的地位,用户通过智能语音助手,可以控制其他设备。因此一旦智能语音设备被攻击,物联网系统的安全就会受到威胁。麦克风与扬声器作为智能语音设备中的重要输入输出传感器,接收与处理的信号主要是机械波。在无法察觉的情况下,机械波中容易嵌入恶意信息,从而威胁智能语音系统的安全。

基于机械波的攻击有多种手段。第一种是超声波攻击,攻击者将正常频率范围的语音信号调制成人体无法听到的超声波信号,然后使用超声波发射器发射调制后的超声波信号,从而实现对智能语音设备的攻击。这种攻击利用了麦克风设备本身的非线性,将隐藏在超声波信道中的语音命令恶意注入麦克风,由于麦克风会将这种超声波识别成正常语音,智能语音设备会被恶意控制。第二种是振动攻击,攻击者将承载智能语音设备的固体作为媒介,通过固体振动的频率来传输语音命令,从而实现对智能语音设备的控制。

\subsubsection{\textcolor{myblue}{\textbf{2. 基于电磁波的攻击 }}}
除了使用机械波进行交互的设备之外,还有一部分设备使用电磁波进行通信。无线路由器可以使用Wi-Fi信号实现信息的交换,毫米波雷达可以使用毫米波信号实现距离的测量与定位。这类主动式传感设备在使用电磁波的信号完成本职工作的同时,信号中还会嵌入所处环境下的额外信息。

攻击者有多种基于电磁波的攻击手段。首先是Wi-Fi攻击,攻击者使用Wi-Fi信号感知人体口腔运动的细粒度变化,并通过机器学习的方法从中分析出语音的内容,从而实现对人体的窃听。然后是毫米波攻击,攻击者使用毫米波信号感知液晶显示器的状态变化,建立模型分析电子屏幕的内容,从而实现对电子屏幕的监测。

\subsection{系统级固件脆弱性}
\label{firmware}
系统级固件的脆弱性,普遍存在于各种计算设备上,包括超级计算机,个人计算机,以及各种物联网设备。从人们意识到固件的脆弱性开始,不断有一些防御的方案被提出,个人计算机和超级计算机越来越被难以攻破,但是对于物联网设备而言,由于种种局限性,目前系统级的固件脆弱性仍是较大的安全威胁。

物联网设备的固件局限性体现在多方面,第一,由于对固件调试以及检测的有效工具较为缺乏,所以有大量携带漏洞的固件存在于实际产品;第二,由于计算能力,内存资源等限制,许多防御的机制例如数据执行保护,地址随机化等无法部署,缺乏安全保护机制的情况下,漏洞可以轻易地被攻击;第三,由于物联网设备固件大多是由第三方设计开发,不会考虑对固件代码的更新,所以即使有漏洞被发现,大量已经部署的设备并不能自动更新代码以修复漏洞。

总的来说,由于缺乏固件调试工具,所以实际产品中存在大量携带漏洞的固件,并且由于第三方不提供更新固件代码服务,即使漏洞被发现,实际产品的固件也不能被修复,最后由于物联网设备本身计算能力和内存资源的局限性,无法部署安全保护机制,导致漏洞可以被轻易利用,进而导致设备本身被成功攻击,引发一系列安全问题。下面我们介绍一些常见的固件相关的漏洞和威胁。


\subsubsection{\textcolor{myblue}{\textbf{1. 内存漏洞 }}}
固件内存漏洞一般由编写代码时的错误引起,会导致对内存的非法访问、控制流劫持等攻击。由于物联网设备固件主要由较为底层语言(例如C语言)编写,在开发过程中会不可避免地引入编码缺陷。硬件层面,物联网设备具有异构性,多样性等特点,使得对固件程序开展规模化和自动化的漏洞检测十分困难;软件层面,由于有限的硬件资源,导致必要的防御机制无法被部署,攻击者更加容易利用内存漏洞\cite{SecSurvey}。

当内存漏洞被发现时,攻击者有多种利用手段。第一种是代码注入\cite{InjectRobot},攻击者可以利用内存漏洞,将一些关键代码注入到内存数据区,进而使漏洞程序运行攻击者设计的代码,通常在个人计算机上,有数据不可执行等防御措施,可以很好地对代码注入进行防护;第二种是控制流劫持\cite{muRAI},攻击者通过利用内存漏洞,可以修改函数返回地址,并且在堆栈上任意构造函数输入,进而劫持目标程序的控制流,通常在个人计算机上,有地址随机化,控制流完整性保护等防御措施,可以对控制流劫持进行保护。此外,攻击者也可以结合其他攻击手段,比如通过对固件二进制进行逆向工程,得到一些关键的敏感控制数据,然后结合得到的控制数据和内存漏洞,对程序权限进行修改,达到攻击目的。

\subsubsection{\textcolor{myblue}{\textbf{2. 逻辑漏洞 }}}
固件逻辑漏洞也是一种常见的固件漏洞,指的是固件在认证、授权、应用功能等方面的设计或实现缺陷。通常,攻击者会利用逆向工程的手段,从固件二进制代码中寻找逻辑漏洞,当攻击者找到代码的逻辑缺陷后,只需要构造特定的输入,就可以使得程序在正常运行的同时,功能发生偏移。比如认证绕过漏洞\cite{PriviSep},就是一个典型的固件逻辑漏洞类型,攻击者通过这种漏洞来绕过系统对于特权指令的权限检查,来实施非法恶意的行为;在智能网络打印机固件程序中的逻辑设计缺陷,在实际的办公场景中会造成任务篡改、机密泄露等后果\cite{ExpPrinter}。


\subsection{前沿研究}

\subsubsection{\textcolor{myblue}{\textbf{1. 侧信道威胁 }}}
无线电通信技术广泛应用在物联网系统当中,而无线电除了能够传输数据信息,在传输过程中还会与物体产生反射,从而携带了部分额外的信息。Wi-Fi信道作为在我们日常生活中不可或缺的信道,其安全性值得被探索与研究。目前许多的工作使用Wi-Fi信号实现了对人体的窃听、运动监测、手势识别、与定位。WiTrack\cite{witrack}利用从用户身上反射的WiFi信号,追踪用户在三维空间中的运动,并提供对人身体各个部位的粗略追踪。WiHear\cite{wihear}依据人体在说话过程中,Wi-Fi信号能够感知到口腔特定的运动轮廓的特点,使用信号处理技术捕捉Wi-Fi信号中人体嘴部产生的微小反射,并利用机器学习的手段,将Wi-Fi信号中的隐藏的口腔运动特征识别成对应的音节,从而实现对人体的窃听。这类基于Wi-Fi信号的侧信道攻击,不容易受到障碍物的影响,而且由于目前Wi-Fi信号广泛被部署,往往无需额外配置设备,难以防范。

激光作为另一种重要的电磁波,在通信、工业、军事等领域发挥了巨大的作用,而目前出现了许多利用激光信号实现的侧信道攻击工作。LightCommands\cite{lightcommands}依据麦克风会对光信号做出反映的特点,通过调节激光的振幅,远距离向麦克风注入激光信号来模拟声音,从而影响智能语音设备,实现对语音系统的控制。Rolling Colors\cite{rolling colors}针对CMOS相机设备的固有弱点,在相机捕捉的图像中注入特殊的图像模式,从而干扰交通灯识别系统的准确性,威胁自动驾驶车辆的行驶安全。


\subsubsection{\textcolor{myblue}{\textbf{2. 固件脆弱性 }}}

物联网设备多种多样,其中工业机器人是一类比较典型的设备。工业机器人在工业上有着大量的应用,并且有着很高的市场市值,一般地,它是由软件控制,与物理世界进行交互的。相关研究\cite{InjectRobot}表明,由于计算机连通性的不断增强,本应该在网络独立环境下工作的工业机器人,基本上被暴露在外界网络,但是在工业机器人重要性与日俱增的同时,其背后的软件实现却增加了安全性的隐患。在现实工业环境中,当前许多种工业机器人会通过FTP服务器暴露在互联网。此时,通过互联网远程对工业机器人背后的控制软件进行侵入,进而影响工业机器人与物理世界的交互行为,最终可以实现攻击操作员或是工业机器人附近人类。也就是说,通过攻击本身具有与物理世界交互能力的物联网设备其背后的软件漏洞,可以造成对物理世界的真实威胁。

其他最新的研究更多地集中在防御上,人们试图通过编译器来保护物联网设备上的裸机程序,或其他手段来帮助检测固件漏洞。裸机的物联网设备,由于没有操作系统来管理权限和硬件资源,这导致攻击者一旦攻破程序本身,就可以直接拥有对各种硬件资源的访问权限,这是裸机程序缺乏权限管理的直接后果。一方面,人们试图提出一种新的基于LLVM的编译器\cite{PriviSep},在编译时识别需要权限的指令,并加以保护。另一方面,人们希望通过在基于微控制器的物联网设备固件上增加返回地址的完整性保护\cite{muRAI},以防御传统的控制流劫持攻击。











