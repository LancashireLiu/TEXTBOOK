% ===========================================================================================
% 我自己定义的一些命令
% ===========================================================================================


% For using the font of \mathpzc
\DeclareMathAlphabet{\mathpzc}{OT1}{pzc}{m}{it}

% 一些数学符号,如箭头等,与两边符号的间距过大,在这里定义比较紧一些间距的符号,这些命令只能用于数学模式
\newcommand {\tleq}{\negthickspace\leq\negthickspace}           % for change the space before and after \leq
\newcommand {\tsubseteq}{\negthickspace\subseteq\negthickspace} % for change the space before and after \subsetq
\newcommand {\tto}{\negthickspace\to\negthickspace}             % for change the space before and after \to
\newcommand {\tin}{\negthickspace\in\negthickspace}             % for change the space before and after \in
\newcommand {\ttimes}{\negthickspace\times\negthickspace}       % for change the space before and after \times
\newcommand {\tplus}{\negthickspace +\negthickspace}            % for change the space before and after +
\newcommand {\tTo}{\negthickspace\Rightarrow\negthickspace}     % for change the space before and after \Rightarrow
\newcommand {\tvdash}{\negthickspace\vdash\negthickspace}       % for change the space before and after \vdash

% 定义一些常用串的组合,这些串只能用于数学模式。定义这些串可减少输入时间
\newcommand {\sembracket}[1]{\llbracket#1\rrbracket}            % for giving the semantic bracket such as [[ _ ]]
\newcommand {\prodmedt}[1]{\langle#1\rangle}                    % for giving the product mediate morphism, such as < _ >
\newcommand {\coprmedt}[1]{[#1]}                                % for giving the product mediate morphism, such as [ _ ]
%\newcommand {\finalhomo}{\text\textexclamdown}                  % for down !
\newcommand {\finalhomo}{\text{!}}                            % use ! for final homomorphism
\newcommand {\eqdef}{\stackrel{\mathrm{def}}{=}}                % for `def' over =

% 为范畴和函子定义一些常用的串,这些串都只能用于数学模式,定义这些串除了减少输入时间之外,
% 也可方便地修改这些符号所使用的字体


% 范畴所用的字体
\newcommand {\Cat}[1]{\mathcal{#1}}                             % #1 is a category,这里#1只能有一个字母,通常是泛指的范畴
\newcommand {\Cats}[1]{\mathpzc{#1}}                            % #1 is a category, 这里#1有多个字母,通常是具体的范畴
\newcommand {\CatC}{\Cat{C}}                                    % Category C
\newcommand {\CatD}{\Cat{D}}                                    % Category D
\newcommand {\CatE}{\Cat{E}}                                    % Category E
% 函子所用的字体
\newcommand {\Fun}[1]{\textsl{#1}}                              % #1 is a functor
\newcommand {\FunT}{\Fun{T}}                                    % Functor T
\newcommand {\FunF}{\Fun{F}}                                    % Functor F
\newcommand {\FunG}{\Fun{G}}                                    % Functor G
\newcommand {\FunH}{\Fun{H}}                                    % Functor H
% 对象类和射类的名字所用的字体,例如Mono(C),代表范畴C的所有单射构成的类等,这里Mono就是这一类射的名字
\newcommand {\Mors}[1]{\mathbf{#1}}                             % 主要是针对各种单射和满射构成的类
\newcommand {\Objs}[1]{\mathbf{#1}}                             % 主要是针对子对象、商对象和关系


% 定义一些常用的设置文本颜色命令
\definecolor{darkgreen}{rgb}{0.00,0.50,0.25}
\definecolor{darkyellow}{rgb}{0.50,0.50,0.00}

\newcommand {\redtext}[1]{\textcolor{red}{#1}}                  % set the red color text
\newcommand {\bluetext}[1]{\textcolor{blue}{#1}}                % set the blue color text
\newcommand {\greentext}[1]{\textcolor{darkgreen}{#1}}          % set the dark green color text
\newcommand {\yellowtext}[1]{\textcolor{darkyellow}{#1}}        % set the dark green color text
\newcommand {\magentatext}[1]{\textcolor{magenta}{#1}}          % set the dark green color text

\newcommand {\define}[1]{{\heiti\redtext{#1}}}                  % 定义中所要定义的概念使用红色黑体字标出
\newcommand {\notice}[1]{{\heiti\bluetext{#1}}}                 % 正文中要注意的句子使用蓝色黑体字标出

\definecolor{mathbgc}{rgb}{1.00,1.00,0.80}                      % 用于数学公式背景色
\definecolor{mathfgc}{rgb}{0.68,0.00,0.00}                      % 用于数学公式前景色
\definecolor{progfgc}{rgb}{0.00,0.00,1.00}
\definecolor{progbgc}{rgb}{0.90,0.90,1.00}
\definecolor{itemiclr}{rgb}{0.00,0.00,0.5}
\definecolor{itemiiclr}{rgb}{0.00,0.39,0.39}
\definecolor{itemiiiclr}{rgb}{0.3,0.00,0.22}
\definecolor{textfgc}{rgb}{0.00,0.25,0.00}
\definecolor{textbgc}{rgb}{1.00,1.00,0.80}
\definecolor{diagfgc}{rgb}{0.45,0.15,0.00}
\definecolor{diagbgc}{rgb}{1.00,0.90,0.80}


\definecolor{formbgc}{rgb}{0.90,0.90,1.00}                      % 用于数学公式背景色
\definecolor{formfgc}{rgb}{0.75,0.00,0.0}                      % 用于数学公式背景色

\newcommand {\formblock}[1]{\colorbox{formbgc}{\textcolor{formfgc}{#1}}}        % 带颜色的公式块

\newcommand {\mathcolorbox}[2]{
\begin{center}
\colorbox{mathbgc}{\textcolor{mathfgc}{
    \begin{minipage}{#1}\vspace{-1.5em}
        #2
    \end{minipage}
}}
\end{center}
}

\newcommand {\formcolorbox}[2]{
\begin{center}
\colorbox{formbgc}{\textcolor{formfgc}{
    \begin{minipage}{#1}\vspace{-1.2em}
        #2
    \end{minipage}
}}
\end{center}
}

\newcommand {\progcolorbox}[2]{
\begin{center}
\colorbox{progbgc}{\textcolor{progfgc}{
    \begin{minipage}{#1}\vspace{-1.5em}
        #2
    \end{minipage}
}}
\end{center}
}

\newcommand {\textcolorbox}[2]{
\begin{center}
\colorbox{textbgc}{\textcolor{textfgc}{
    \begin{minipage}{#1}
        #2
    \end{minipage}
}}
\end{center}
}

\newcommand {\truek}{~\mathbf{true}~}
\newcommand {\falsek}{~\mathbf{false}~}
\newcommand {\skipk}{~\mathbf{skip}~}
\newcommand {\ifk}{~\mathbf{if}~}
\newcommand {\thenk}{~\mathbf{then}~}
\newcommand {\elsek}{~\mathbf{else}~}
\newcommand {\fork}{~\mathbf{for}~}
\newcommand {\whilek}{~\mathbf{while}~}
\newcommand {\dok}{~\mathbf{do}~}
\newcommand {\tok}{~\mathbf{to}~}
\newcommand {\downtok}{~\mathbf{downto}~}

%\newcommand {\undef}{\textsf{undefine}}

\newcommand {\dfracnoline}[2]{\genfrac{}{}{0pt}{0}{#1}{#2}}


\newcommand {\vskipa}{\vspace{+0.2em}}
\newcommand {\vskipb}{\vspace{+0.3em}}
\newcommand {\vskipc}{\vspace{+0.5em}}
\newcommand {\vskipd}{\vspace{+0.8em}}


\newcommand {\explain}[1]{//~\mbox{\bluetext{#1}}}   % 用于数学模式下的注释

\newcommand {\fcoalgsn}{\mathbb{R}^\mathbb{N}}
\newcommand {\true}{\textmd{tt}}
\newcommand {\false}{\textmd{ff}}
\newcommand {\coalg}[1]{~\mathfrak{#1}}
\newcommand {\fsym}[1]{\mathrm{#1}}
\newcommand {\sgto}[1]{\rightsquigarrow^{#1}}
\newcommand {\bisim}{\sim}

\newcommand \utype{\mathsf{Unit}}
\newcommand \stype{\mathsf{State}}
\newcommand \uterm{\mathsf{unit}}
\newcommand \sterm{\mathsf{state}}
\newcommand \fsterm{\mathsf{fst}}
\newcommand \snterm{\mathsf{snd}}
\newcommand \ilterm{\mathsf{inl}}
\newcommand \irterm{\mathsf{inr}}
\newcommand \csterm{\mathsf{case}}
\newcommand \tterm{\mathsf{trans}}
\newcommand {\spec}[1]{\textsf{spec}(#1)}
\newcommand \FunS{\Fun{S}}

\newcommand \spk[1]{\textbf{#1}}                                % 规范中的关键字的字体
\newcommand \atk[1]{\textsf{#1}}                                % 规范中断言中的关键字的字体
\newcommand \spm[1]{\textsf{#1}}                                % 规范中的操作名的字体
\newcommand \spc[1]{\textrm{#1}}                                % 规范中其他名称(包括注释)的字体
\newcommand \scase{\atk{CASES}}
\newcommand \secas{\atk{ENDCASES}}

\newcommand \dfraciff[2]{\xlongequal[{\displaystyle #2}]{\displaystyle #1}}

\newcommand \tvalue{\mathbf{1}}
\newcommand \fvalue{\mathbf{0}}
\newcommand \ran{\mathbf{ran}}
\newcommand \funid{\mathbf{id}}
\newcommand \circledlabel[1]{\scriptsize\textcircled{\tiny #1}}
